% Created 2021-06-22 Tue 20:36
% Intended LaTeX compiler: pdflatex
\documentclass[11pt]{article}
\usepackage[utf8]{inputenc}
\usepackage[T1]{fontenc}
\usepackage{graphicx}
\usepackage{grffile}
\usepackage{longtable}
\usepackage{wrapfig}
\usepackage{rotating}
\usepackage[normalem]{ulem}
\usepackage{amsmath}
\usepackage{textcomp}
\usepackage{amssymb}
\usepackage{capt-of}
\usepackage{hyperref}
\date{\today}
\title{Daily Quiet Time Bible Study}
\hypersetup{
 pdfauthor={},
 pdftitle={Daily Quiet Time Bible Study},
 pdfkeywords={},
 pdfsubject={},
 pdfcreator={Emacs 27.2 (Org mode 9.4.4)}, 
 pdflang={English}}
\begin{document}

\maketitle
\tableofcontents


\section{June 22, 2021}
\label{sec:org009a16c}

\textbf{\textbf{JOHN 8:12-59: JESUS, THE LIGHT OF THE WORLD}}

Jesus never spoke in public without creating controversy. In fact, he
was constantly in trouble! Rather than retreating behind the safety
of a pulpit, Jesus spoke in settings where people were bold enough to
talk back. In this portion of John's story, Jesus makes a series of
claims about himself. Each claim is met by a challenge from his
enemies. Each challenge is then answered, and the answer leads to the
next claim. Throughout this interchange, Jesus shows us how to speak
the truth in the face of hostility. He also reveals some amazing
things about himself.


\subsection{Warming Up to God}
\label{sec:orgc34abf3}


\textbf{\textbf{Have you ever tried to talk about Christ with a family member or coworker who was hostile to your message? How did you feel at the time? Thank God for giving you a Savior who understands everything we experience.}}

Jesus himself experienced people who were hostile to his message. They
wanted to kill him for his message. They misunderstood him
completely. We can face the same today in our witness of Christ. Our
master has experienced it all.

Read John 8:12-59

\subsection{Discovering the Word}
\label{sec:org50ced80}

\textbf{\textbf{The Pharisees challenge the validity of Jesus’ claim (v. 13). How does Jesus answer their challenge (vv. 14-18)?}}

Jesus' claims were challenged as he was the only witness to it. Jesus answered back to their challenge:

\begin{enumerate}
\item He knows where he came from and where he was going, but the Pharisees have no idea about it.

\item They judge by human standards but Jesus was not judging anyone.

\item Judging by human standards, there have to be two witnesses. Jesus has two witnesses: himself and his father. So his claims are valid even by their standards.
\end{enumerate}


\textbf{\textbf{Jesus' reference to his Father leads to his second claim—that he came from God. How does this claim heighten the tension between Jesus and the Jews (vv. 19-30)?}}

There was further dispute about Jesus. They asked him who he was. He told them about how he had come from God and that he spoke only what the Father told him to say so.

\textbf{\textbf{Jesus makes another startling claim in verses 31-32: "If you hold to my teaching . . . then you will know the truth, and the truth will set you free." Why does holding to Jesus’ teaching lead to true knowledge and freedom?}}

Because Jesus was from the Father and his teachings are not from man. So if we hold to his teachings, we will know the truth and the truth will set us free.

\textbf{\textbf{Jesus' opponents also claim to have both Abraham and God as their father. According to Jesus, how does their conduct contradict their claim (vv. 39-47)?}}

Since they were looking for ways to kill him they dont know God. Had they known God they would have loved Jesus.


\subsection{Applying the Word}
\label{sec:orgce78a36}

\textbf{\textbf{Why is our conduct the truest test of our beliefs?}}

We act according to what we believe. A person who does not believe in God will act differently from a person who believes in God. Even if we claim to know God, our actions will tell us who we really are.

\textbf{\textbf{In what ways does your lifestyle validate (or invalidate) your claim to be a follower of Christ?}}

In various situations, our response should be different from how non-belivers act. That validates our claim to be followers of Christ.

\subsection{Responding in Prayer}
\label{sec:orgfb07818}

\textbf{\textbf{Ask God to help you change the parts of your life that don't match your beliefs.}}

Lord, help me to live my life worthy of my faith in you. Amen.


\section{June 21, 2021}
\label{sec:org5ad3a2d}

\textbf{\textbf{1 TIMOTHY 2: BARRIERS TO RENEWAL}}

For more context before you begin studying, read this \href{https://www.ivpress.com/daily-bible-study/introducing-1-timothy}{introduction to the book of 1 Timothy.}

Do you ever have difficulty approaching God in prayer? The Christians
in Ephesus did. The barriers to prayer described here are anger, an
over-emphasis on appearance, and an inappropriate role for
women. Having sized up the situation and reminded Timothy of his
mission, Paul outlines first steps to dealing with the needs of the
church.

\subsection{Warming Up to God}
\label{sec:org2f4a0fe}

When do you find prayer difficult?

\href{https://www.biblegateway.com/passage/?search=1\%20Timothy\%202\&version=NIV\&interface=print}{Read 1 Timothy 2}

\subsection{Discovering the Word}
\label{sec:org1a6dc77}

\textbf{\textbf{Find all the times Paul says "all" and "everyone" (vv. 1-6). What does the use of these terms communicate about God?}}

Shows God loves all people:
\begin{itemize}
\item first of all
\item all people
\item all those in authority
\item in all godliness and holiness
\item all people to be saved
\item ransom for all people
\end{itemize}

\textbf{\textbf{Why do you think Paul emphasizes the word one in verse 5?}}
\begin{itemize}
\item He is speaking of the uniqueness of Christ - as the only mediator between God and mankind
\end{itemize}

\textbf{\textbf{This entire chapter deals with worship. The church in Ephesus was probably a network of house churches. Their worship may have been patterned after the Jewish synagogues which separated men and women. What problem hindered the worship by men (v. 8)?}}
\begin{itemize}
\item There was anger and dispute among them
\end{itemize}

\textbf{\textbf{What problem hindered the worship by women (v. 9)? 5. In verses 11-12, Paul forbids women to teach men. But in 1 Corinthians 11:5, he tells them how to dress when they preach (or prophesy). How do you reconcile these texts?}}
\begin{itemize}
\item The women were more concerned about physical beauty than they ought to be as someone who professed to be worshiper of God.
\end{itemize}

\subsection{Applying the Word}
\label{sec:orgc0df1c2}

\textbf{\textbf{According to this passage, what could hinder worship and sharing the gospel?}}
\begin{itemize}
\item Anger and dispute within the church could hinder worship and sharing the gospel
\end{itemize}

\textbf{\textbf{Consider which of those are problems for you. How can you better deal with them?}}
\begin{itemize}
\item Yes there are such problems in my church too. There are disputes and anger which are concern in my church today.
\end{itemize}

\subsection{Responding in Prayer}
\label{sec:org984f98f}

\textbf{\textbf{Pray for your church's worship and for your personal worship.}}
\begin{itemize}
\item Lord, I pray for my church that there will be unity and oneness just as you are one. For your glory in Jesus name. Amen
\end{itemize}


\section{June 20, 2021}
\label{sec:org5818e03}

\textbf{\textbf{JOHN 7:53—8:11: CAUGHT IN ADULTERY}}

Nothing is more humiliating than being caught in an act of
disobedience! Whether it's a child with his hand in the cookie jar or
an adult driving over the speed limit, we all know the sinking
feeling of being caught. In John 8, a woman is caught in the most
awkward of situations—in the very act of adultery. The way Jesus
responds to her may surprise you.

\subsection{Warming Up to God}
\label{sec:org932e709}

Think of a time when you hurt someone and that person was willing
to forgive you. How did it feel to be forgiven? Thank God for
extending forgiveness to you.

Read John 7:53—8:11

\subsection{Discovering the Word}
\label{sec:org067458f}

\textbf{\textbf{What do we know about the character and motives of those who bring this woman to Jesus?}}

They were a bunch of self-righteous people who think the woman caught
in adultery was greater sinner than them. Even worse is the fact that
they used her to trap Jesus.

\textbf{\textbf{While it is obvious that the woman is guilty, what elements of injustice can you find in this situation?}}

If they thought that she was guilty and deserved to be stoned to death
as the Law of Moses required, there was no reason to bring her to
Jesus. They could have executed her themselves. They somehow want to
put the blame on Jesus or put him in a difficult situation.

\textbf{\textbf{The Pharisees and teachers were often very self-righteous. Why do you think they went away rather than stoning the woman (vv. 7-9)?}}

They went away because non of them was without sin. In today's
culture, the people might have still pressed for some other means to
trap Jesus. Here the people left, which is better than the world today
(my opinion).

\textbf{\textbf{How would you describe Jesus' attitude toward the woman (vv. 10-11)?}}

Jesus loves the sinner. His sinless, he hates sin but not the sinner. He wants her to sin no more.

\subsection{Applying the Word}
\label{sec:orgdcecd09}

\textbf{\textbf{What can we learn from this passage about Christ’s attitude toward us—even when we feel awful about ourselves?}}

Jesus is friend of sinners. I can be really sure about his love for a
sinner like me. We are to leave our way of sin.

\textbf{\textbf{What does it teach us about forgiving and accepting others?}}

Just as Jesus don't teach about punishing the sinner, it is also our
duty not to judge others. We should look at our own lives. We are to
love others just as he loved us.

\textbf{\textbf{Who do you need to offer your forgiveness to?}}

I need to forgive people who have "wronged" me.

\subsection{Responding in Prayer}
\label{sec:org3c3ff2b}

\textbf{\textbf{Ask God to show you what it means to forgive.}}
\section{June 18, 2021}
\label{sec:org19eea4b}

\textbf{\textbf{JOHN 7:1-52: CONFUSION OVER CHRIST}}

Not long ago I had a series of conversations with a young man about
Jesus Christ and why faith in him is so important. At first, the
young man was interested. He was open to listen to God’s Word and to
consider Christ’s claims. As time went on, however, he became more
and more hostile to Christ. Finally, he told me that he didn’t want
to pursue his investigation any further. He had decided to reject
Christ and his offer of salvation.

That is precisely the pattern that John traces in his Gospel. In the
early chapters, men and women responded to Jesus with belief. Then
some of those who were following him turned away. Now open warfare
breaks out between Jesus and his enemies—and yet, some still seek the
truth. This chapter will help you respond positively to the wide
variety of attitudes toward Jesus today.

\subsection{Warming Up to God}
\label{sec:org338a2ba}

"If anyone is thirsty, let him come to me and drink." These are
Jesus’ words to us today. In what way do you need Jesus’ spiritual
refreshment? Be quiet before him and experience the "streams of
living water."

Read John 7:1-52

\subsection{Discovering the Word}
\label{sec:orgdb19f1f}

\textbf{\textbf{The first blast of hostility against Jesus comes from his own family (vv. 1-13). How would you characterize the statements made by Jesus' brothers (vv. 3-5)?}}

\textbf{\textbf{When Jesus makes his presence in Jerusalem known, people begin to challenge the origin (and, therefore, the authority) of his teaching. According to Jesus, how can we verify the truth of his teaching (vv. 16-18)?}}

\textbf{\textbf{What other opinions or questions do people have about Jesus in verses 20-36?}}

\textbf{\textbf{How does Jesus respond to each one?}}

\textbf{\textbf{On the last day of the Feast of Tabernacles, large vats of water were poured out on the pavement of the temple court as a reminder of God's provision of water in the wilderness. With that custom in mind, how would you explain the significance of Jesus' remarks in verses 37-39?}}

\subsection{Applying the Word}
\label{sec:org13837e3}

\textbf{\textbf{What counsel would you give a believer who faces spiritual opposition from his or her family?}}

\textbf{\textbf{Which of the opinions about Jesus you have identified in this chapter are still expressed today, and in what way?}}

\textbf{\textbf{Based on Jesus' example, what should our response be to such reactions?}}

\subsection{Responding in Prayer}
\label{sec:org41ac020}

\textbf{\textbf{Pray for wisdom as you face various reactions to Jesus.}}

\section{June 17, 2021}
\label{sec:orgd77c67a}

\textbf{\textbf{JOHN 6: JESUS, THE BREAD OF LIFE}}

Do you realize that during your lifetime you will probably spend over
thirty-five thousand hours eating? That's the equivalent of eight
years of nonstop meals, twelve hours a day! The problem, of course,
is that even after a big meal we get hungry again. At best, food only
satisfies us for a few hours.

Yet in this chapter, Jesus offers us food that satisfies our hunger
forever. You can't buy it in a grocery store. It is found only in
Jesus himself.

\subsection{Warming Up to God}
\label{sec:orga2f62fc}

\textbf{\textbf{How do you usually respond to an "impossible" situation—a problem in your life that doesn't seem to have a solution?}}

In such situations, I have prayed to God and help me get through. I
was very disappointed once that my prayers were not answered (as I
thought). I questioned a lot about life, God and his will. It was not
easy having been through. I thought I could have just helped myself
instead of asking God to help me. It made me very angry with everything. 

\href{https://www.biblegateway.com/passage/?search=John\%206\&version=NIV\&interface=print}{Read John 6}

\subsection{Discovering the Word}
\label{sec:org0e9c6a2}

\textbf{\textbf{Read John 6:1-15. How would you characterize Philip's and Andrew's response to the problem of feeding this enormous crowd (vv. 5-9)?}}

Philip was practical in his response by saying that it will cost half a year's wage to give a bite to the people.

Andrew offered a very small solution that there was a small boy with five loaves and two fish, but impossible for the people to be fed.

Both knows the situation is impossible. Andrew goes a little bit further in offering something rather than nothing.


\textbf{\textbf{Imagine that you are one of the disciples, rowing the boat in dark, rough waters (vv. 16-21). How would your concept of Jesus have been altered by seeing him walk on water?}}

\textbf{\textbf{The next day the people were hungry again, so they came seeking Jesus (vv. 22-25). How does he try to redirect their thinking (vv. 26-33)?}}

\textbf{\textbf{Based on the remarks of some in the crowd (vv. 41-42), do you think they finally understood what Jesus was saying? Explain.}}

\textbf{\textbf{When Jesus said, "This bread is my flesh," the crowd could think only of cannibalism (v. 52). What do you think it means to eat Jesus' flesh and drink his blood (vv. 53-59)? Is this something we do once for all time, or is it an ongoing process? Explain.}}

\textbf{\textbf{In verses 60-71 Jesus turns away from the crowd and focuses on his disciples. How would you describe their responses to his "hard teaching"?}}

\subsection{Applying the Word}
\label{sec:org444588d}

\textbf{\textbf{Which response in question 6 best describes your present attitude toward Jesus? Explain.}}

\textbf{\textbf{Jesus has contrasted the two appetites found in every person—the appetite for food that perishes and the appetite for food that endures. In what ways has Jesus satisfied the spiritual hunger in your heart?}}

\subsection{Responding in Prayer}
\label{sec:org113dc4d}

\textbf{\textbf{Praise God for satisfying all your needs.}}

\section{June 16, 2021}
\label{sec:org17e6dd7}

\textbf{\textbf{JOSHUA 24: WHO WILL SERVE THE LORD?}}

Throughout the book of Joshua the Lord has demonstrated his
faithfulness and his power. Every promise he made was fulfilled; every
battle he fought was won. Now at the conclusion of the book he asks
Israel and us to reaffirm the most important decision of our lives:
"Choose for yourselves this day whom you will serve" (v. 15).

\subsection{Warming Up to God}
\label{sec:org5d414ea}

God has given us the ability to make choices. What are some of the
inherent benefits and dangers of this freedom for you?

\href{https://www.biblegateway.com/passage/?search=Joshua\%2024\&version=NIV\&interface=print}{Read Joshua 24}

\subsection{Discovering the Word}
\label{sec:orgdefae58}

\textbf{\textbf{Why do you think the Lord recounts Israel's history from beginning to end (vv. 1-13)?}}
\begin{itemize}
\item It recounts Israel's journey from their ancestor of long ago,
Abraham to the day they possessed the Promised Land.
\item How God has been faithful in keeping his promises to Abraham, Issac
and Israel
\end{itemize}

\textbf{\textbf{Joshua and the Israelites repeat the word serve thirteen times in verses 14-27. Why is this a good word to describe our duty to God?}}
\begin{itemize}
\item Either we serve the Lord or serve someone/something else. It is ours
to choose whom we serve.
\item Our duty is to serve and obey the Lord. We are his servants. What
better word than "serve" to describe our duty to Him.
\end{itemize}

\textbf{\textbf{Why might serving the Lord seem undesirable to the Israelites (v. 15)?}}
\begin{itemize}
\item Serving the Lord might seem undesirable as the grass seems to be
greener on the other side. When Israel sees the gods of their past
and of their present neighbors, they might think it seems more
desirable to serve other gods.
\end{itemize}

\textbf{\textbf{Why doesn't Joshua accept the Israelites' first pledge to serve the Lord (v. 19)?}}
\begin{itemize}
\item Joshua is reminding them that serving the Lord is not easy. In fact
we are not able to since He is a holy God, jealous God and will not
forgive our rebellion and sins if we forsake him.
\item He wants to make sure that the people really meant what they say
about serving the Lord
\end{itemize}


\textbf{\textbf{What is the purpose of the witnesses mentioned in verses 22 and 27?}}
\begin{itemize}
\item Witness is a reminder that we have made promise to God. It helps us
not to stray away from the path we have chosen to take
\end{itemize}

\subsection{Applying the Word}
\label{sec:orgd1e5be9}

\textbf{\textbf{Israel must choose whom they will serve (v. 15). What choices to you have about whom to serve?}}
\begin{itemize}
\item We also have a choice today: money, career, or perhaps
entertainment. We need all these in life, but we can go to the
extent that we value them or keep them above God. And as Christians,
that is not the right choice.
\end{itemize}

\textbf{\textbf{In what ways is serving the Lord difficult for you?}}
\begin{itemize}
\item Serving the Lord is difficult for me due to temptations, the desires
of the flesh and weaknesses.
\end{itemize}


\textbf{\textbf{What makes serving him worthwhile?}}
\begin{itemize}
\item Serving him is most worthwhile because there is nothing greater than
serving God. If we are seeking to find purpose in our lives then, it
is serving God.
\end{itemize}

\subsection{Responding in Prayer}
\label{sec:org270d524}

\textbf{\textbf{Pray that God will give you courage to rise to the challenge of Joshua—to obey God more fully.}}

Lord help me to always remember to serve you. There is nothing worth
more than to serve the living God, the one and only true God.

\section{June 14, 2021}
\label{sec:org5883326}

\textbf{\textbf{JOSHUA 22: WILL THE NATION SURVIVE?}}

For more context before you begin studying, read this \href{https://www.ivpress.com/daily-bible-study/introducing-joshua}{introduction to
the book of Joshua.}

The tribes which had been given land on the east side of the Jordan
had fulfilled their commitment to the rest of Israel. They had entered
the Promised Land with the others and had helped them to fight. Now it
was time for them to return to their own homes. But since the
Israelites would be living on both sides of the river, the Jordan Rift
Valley became a natural barrier which hindered the unity of the twelve
tribes. Could Israel still remain unified? Their response to a
national crisis illustrates principles for maintaining Christian
unity.

\subsection{Warming Up to God}
\label{sec:orgc17e2c0}

God is glorified by your praise and worship. Spend some time offering
your thanks for who he is.

\href{https://www.biblegateway.com/passage/?search=Joshua\%2022\&version=NIV\&interface=print}{Read Joshua 22}

\subsection{Discovering the Word}
\label{sec:org2cf998c}

\textbf{\textbf{Why does Joshua commend the tribes of Reuben, Gad and Manasseh (vv. 1-4)?}}
\begin{itemize}
\item They were commended for doing all that Moses had commanded them to do in carrying out the mission the Lord had given them (to conquer the land).
\end{itemize}

\textbf{\textbf{What blessings had these tribes received from God because of their faithful service (vv. 6-9)?}}
\begin{itemize}
\item They returned to their homes with great wealth
\end{itemize}

\textbf{\textbf{Why were the western tribes prepared to do battle over the issue of the altar (vv. 10-20; see also Deut 12:1-14)?}}
\begin{itemize}
\item They thought the tribes across the Jordan had rebelled against the Lord by building an altar for themselves. The Lord had not tolerated sin in the lives of Israel. So they wanted to root out the evil from their midst.
\end{itemize}


\textbf{\textbf{What reasons did the eastern tribes have for building the altar (vv. 21-29)?}}
\begin{itemize}
\item The eastern tribes replied that their intention was quite the opposite of building an altar for themselves. The altar was to be a reminder to them and the western tribes that the eastern tribes also have a share in the Lord. The altar was a replica of the Lord to stand as witness that they also have share in the Lord.
\end{itemize}

\textbf{\textbf{How did both sides in this dispute show that they were honoring God (vv. 30-34)?}}
\begin{itemize}
\item This incident shows that both sides are jealous for the Lord. Both their intentions are to follow the commandments of the Lord.
\end{itemize}

\subsection{Applying the Word}
\label{sec:orge9f8055}

\textbf{\textbf{What can we learn from this story about confronting those whose actions seem offensive?}}
\begin{itemize}
\item We should first learn from the other person before coming to a conclusion. The intentions of the other party could be quite good though it may appear contrary to us.
\end{itemize}

\textbf{\textbf{What should our attitude be toward those who have misunderstood our actions?}}
\begin{itemize}
\item It is our duty to explain clearly our position and what our actions will achieve when we are misunderstood.
\end{itemize}

\textbf{\textbf{What principles do you find in this chapter for restoring and maintaining unity in your church or fellowship group?}}
\begin{itemize}
\item Two opposing parties or views should see the other's point of view and not take things too personally and resolve conflicts. Everyone working for the Lord have the best intentions for the Lord. If we are wrong, we should also be willing to admit that we have been wrong.
\end{itemize}

\subsection{Responding in Prayer}
\label{sec:org1325fbb}

\textbf{\textbf{Pray that God would bring unity to his church around the world.}}
\begin{itemize}
\item I pray for my church that there will be perfect unity. That all of us will be working for the Lord. Satan should not take control of our thoughts, words and actions to destroy unity in the church.
\end{itemize}


\section{June 12, 2021}
\label{sec:orge2d1c22}

\textbf{\textbf{JOSHUA 20—21: PAKAI IN A THUTEPHO A BULHIT E}}

Na tahsanna chu na duhthusam meimei toh ipi akikhehna um em? Ipi jeh a
Pathen in i taonau sang in te tia ikinep diu ham, ahilouleh i natohnau
ahilouleh i insung u aman khohsah inte itiu ham? I tahsanna uhi golhoi
meimei ahipoi ti hi iti ihet thei diu ham? Pathen lekhabu in asei chu
tahsanna dihtah chu Pathen thutepna a kingam ding ahi ati. Pathen in
thilkhat abol ding a akitep leh, a molso teiding ahi tihi i ginchat
diu ahi. Pathen in aboldia akitep lou leh, i tobang in tahsanna nei
jong leu hen, idei u chu hung umjeng louding ahi. Joshua 20-21 sunga
hi Pathen in Israel te kom a akitepna ho, a chaina chan a ipi hungsoh
am ti imu diu ahi.

\subsection{Pathen kom hin nailut}
\label{sec:orga2408a0}

Phatchesa a Pathen a na kingaina ipi um em? Aman Bible a akitepna toh
kitoh ipi um am?

Joshua 20-21 sim in.

\subsection{Pathen thu kholdoh}
\label{sec:orgc82c194}

\textbf{\textbf{Kiselna ding khopi ho chu ipi jeh a um ham (20:1-6)?}}

Kiselna ding khopi ho chu mikhat in khutsoi a aloi ahilou leh mi khat
atha khah leh a jamlutna ding a um ahi. Kiselna khopi a chu mikhat chu
a lhailut leh, koiman a tha theilou ding ahi.

\textbf{\textbf{Pathen in hitobang kiselna khopi sem uvin tia thupeh anei hin ipi
 eihil uvam?}}

Pathen in toltha bollou din eidei uve. Mikhat in khutsoi a mi khat a
tha khah leh, koiman achung a phu alah ding adeipoi. Mona neilou khat
that chu tolthat ahi. Hitobang ho umlou ding adei ahi.

\textbf{\textbf{Levi phung mite chu phungdang toh akikheh nau ipi ham (see 13:14, 33; 14:3-4; 18:7)?}} 

Levi phung mite chu Pathen kin thempu a kisalel ding ahiuvin,
phungdang toh aki khe uve. Chuleh amaho chu gam kihopna a jong chantum
anei pouve.

\textbf{\textbf{Pathen in Levi mite chu Israel pumpi a thecheh chu ipi thilgon anei hinte nati am?}}

Jacob in Israel sunga kithecheh diu ahi tia jong anasei chu guilhung
hinte tin ka gel e. Ahin thempu dia lhenchom ahinau hi amaho dinga
hamphatna jong ahi tia jong gelthei ahi.

\textbf{\textbf{21:43-45 in Joshua lekhabu a dia a chaina iti asem am?}}

Pathen in a kihahselna bang bangin Israel te chu gam alosah
 tai. Pathen hi akihahselna subulhit Pathen ahi tin a bung hi akichai
 e.

\subsection{Applying the Word}
\label{sec:orgdeb0c25}

\textbf{\textbf{Christian ihi jeh uva, Pathen in eiho chunga ipipi kitepna anei am?}}

Pathen in eiumpi jing diu ahi tin akitem e. A Lhagaon Theng eihin
solpeh diu ahi tin jong asei e. Pathen in a cha in eisim uve. Loupi
jing tahen Pathen chu.

\textbf{\textbf{Pathen in akitepna ho na hinkho a itobang a molso in aum e ti na mu am?}}

Pathen in taona asan ding ahi tin, a thutheng ah mun tamtah a muthei
ahi. Pathen in ka taona asang in, damtheina eipei.


\subsection{Responding in Prayer}
\label{sec:org15b18d1}

\textbf{\textbf{Pathen in kitah tah a akitepna ho a molso jeh in amin thangvah in.}}

Ka kipah e Pakai ka taona na san jal in. Na thutheng a nei umpi jing
ding ahi tia na kitepna bang a nei umpi jing jeh in, Pakai na min ka
choiaan e.
\section{June 12, 2021 dup}
\label{sec:org22e0cd5}

\textbf{\textbf{JOSHUA 20—21: THE LORD FULFILLS HIS PROMISES}}

Na tahsanna chu na duhthusam meimei toh ipi akikhehna um em? Ipi jeh a
Pathen in i taonau sang in te tia ikinep diu ham, ahilouleh i natohnau
ahilouleh i insung u aman khohsah inte itiu ham? I tahsanna uhi golhoi
meimei ahipoi ti hi iti ihet thei diu ham? Pathen lekhabu in asei chu
tahsanna dihtah chu Pathen thutepna a kingam ding ahi ati. Pathen in
thilkhat abol ding a akitep leh, a molso teiding ahi tihi i ginchat
diu ahi. Pathen in aboldia akitep lou leh, i tobang in tahsanna nei
jong leu hen, idei u chu hung umjeng louding ahi. Joshua 20-21 sunga
hi Pathen in Israel te kom a akitepna ho, a chaina chan a ipi hungsoh
am ti imu diu ahi.

How does faith differ from wishful thinking? Why should we expect God
to answer our prayers, to provide us with food and clothing, or to be
involved in jobs or family? How do we know our faith isn't simply
foolishness? Scripture tells us that true faith must be grounded in
God's promises. If God has promised to do something, then we can trust
him wholeheartedly. If he has not promised to do something, then all
the faith in the world won't make it happen. In Joshua 20—21 we see
the final outcome of God's promises to Israel.

Warming Up to God

Think of a time recently when you trusted God for something. Which of
his promises applied to that situation?

Phatchesa a Pathen na kingaina ipi um em? Aman Bible a akitepna toh kitoh ipi um am?

Read Joshua 20-21

Discovering the Word

Kiselna ding khopi ho chu ipi jeh a um ham?

What was the purpose of the cities of refuge (20:1-6)?

Pathen in hitobang kiselna khopi sem uvin tia thupeh anei hin ipi eihil uvam?
What does the command to establish these cities teach us about the Lord?

Levi phung mite chu phungdang toh akikheh nau ipi ham (see 13:14, 33; 14:3-4; 18:7)?
How did the Levites differ from the other tribes (see 13:14, 33; 14:3-4; 18:7)?

Pathen in Levi mite chu Israel pumpi a thecheh chu ipi thilgon anei hinte nati am?
What do you think God's purpose was in scattering the Levites throughout the land?

21:43-45 in Joshua lekhabu a dia a chaina iti asem am?
How does 21:43-45 provide the climax to the book of Joshua?

Applying the Word

Christian ihi jeh uva, Pathen in eiho chunga ipipi kitepna anei am?
What are some of the promises God has made to us as Christians?

Pathen in akitepna ho na hinkho a itobang a molso in aum e ti na mu am?
In what ways have you seen God fulfill these promises in your life?

Responding in Prayer

Pathen in kitah tah a akitepna ho a molso jeh in amin thangvah in.
Praise God for his faithfulness in keeping his promises to us.
\section{June 11, 2021}
\label{sec:org678289e}

\textbf{\textbf{JOHN 4: SOUL \& BODY—SAVINGS \& HEALING}}

"I love humanity; it's people I can’t stand!" Those well-known words
from a member of the "Peanuts" gang still make us chuckle. But our
smiles hide the fact that we sometimes feel exactly like that. John
says very little about Jesus’ contact with the multitudes. But long
sections of the Gospel are devoted to conversations Jesus had with
individuals. In John 4 we see Jesus reach out first to a woman, then
to his disciples, and finally to a grieving father. Watching Jesus
give himself to people with love and compassion will help us care for
those God puts in our paths.

\subsection{Warming Up to God}
\label{sec:orgbe416fe}

When have you recently felt that you were being mobbed by the
multitudes? Ask God to help you to take care of yourself even as you
try to help others.

Read John 4

\subsection{Discovering the Word}
\label{sec:orgdfa0a7d}

What is surprising about Jesus’ question to the Samaritan woman
(vv. 8-9)?
\begin{itemize}
\item Jews don't associate with Samaritans. That's why a Jew (Jesus)
asking a drink from the Samaritan woman was surprising.
\end{itemize}

Why does the woman suddenly change the subject and begin talking
about the controversy over the proper place of worship (vv. 16-20)?
\begin{itemize}
\item From her conversation with Jesus, she came to know something special
about Jesus since he told her about her life

\item Therefore, she changed the topic to spiritual aspects of life

From verses 27-42, do you think the Samaritan woman genuinely
 believed? What do you see in the passage that supports your
 position?
\item She told her townsmen about Jesus. She believed that Jesus is the
Messiah because Jesus said so and also everything about her life.

After his encounter with the Samaritan woman, what specific lessons
 does Jesus apply to his disciples and to us (vv. 34-38)?
\item The harvest is ready, the fields are ripe now
\item The job at hand for the disciples is to go to the field and reap the
harvest

What does the "second miraculous sign" Jesus performs (vv. 43-54)
 reveal about him?
\item He is the Messiah, Saviour of the world.
\end{itemize}

\subsection{Applying the Word}
\label{sec:org1863306}

What has Jesus taught you in this chapter about meeting the specific
needs of those around you?
\begin{itemize}
\item There are people who are neglected and despised around us. As
Christ did, we are to reach out to them
\end{itemize}

What present-day situations might arouse the same racial, religious
and sexual prejudices as the Samaritan woman did?
\begin{itemize}
\item There are many situations based on race, religion and sexual
prejudices in present day people face discrimination
\end{itemize}

How could you reach someone who has been rejected by the world, as
Jesus did?
\begin{itemize}
\item Tell them Jesus loves and cares for them
\end{itemize}

\subsection{Responding in Prayer}
\label{sec:org636c97f}

Ask God to help you be aware of the "Samaritans" around you. Ask him
to help you reach out to them.




\section{June 10, 2021}
\label{sec:org7747e85}

\textbf{\textbf{JOSHUA 14—18: JOSHUA DIVIDES THE LAND}}

Christians often feel more like captives than conquerors. What hinders
our spiritual progress? Why do we sometimes experience so little when
we are promised so much? God had proven himself to Israel throughout
their many battles. He had promised to be with them in the conquest of
Canaan and had kept his promise. Yet in spite of many victories, much
of the land remained to be conquered. These chapters look at why
Israel had failed to possess all that God had promised.

\subsection{Warming Up to God}
\label{sec:orgf1056e0}

In what one area of your life would you most like to see spiritual
progress?

\begin{itemize}
\item Ka lhagao hinkho a ka lhasamna tah khat chu Taona ahi. Bible jong ka
sim thou nai. Ahin, taona hi phatah a nei chu thuchom ahin, anei
mei2 jong ka neijoulou ahi
\item Taona a galsat khat hiding tia ka ki gelhah keo hilou in, kitepna
jong ka nei in, tunichan in ka nei tahtah thei naipoi
\item Aban ah lungthim pumpi a Pathen ngailut tihi ka jou naisai poi
\end{itemize}


\href{https://www.biblegateway.com/passage/?search=Joshua\%2014-18\&version=NIV\&interface=print}{Read Joshua 14-18}

\subsection{Discovering the Word}
\label{sec:org7911ac5}

How is Caleb's faith just as strong at eighty-five as it had been at
forty?
\begin{itemize}
\item Caleb in kum somget le nga jouvin jong, a thahat nalai in, Pathen in
atep peh bang in gamlah be ding anom nalai e.

\item Kum tam leh lhom thu ahipoi. Ama thahat le hatlou thu jong ahipoi. A
Pathen thahatna a akingai jeh ahi.
\end{itemize}

Note 15:63, 16:10 and 17:12-13. God had promised to drive these
Canaanites out of the land. Why then do you think Israel had
difficulty dislodging them (see also Ex 23:29-30 and Dt 7:22-24)?
\begin{itemize}
\item Pathen in a thutepna chu ahileh, a gam a ana chengte chu a nodoh
sohdiu ahi. Ahin khatvei no a nodoh lou diu ahi. Ajeh chu gamsa tam
intin, amaho boina hiding ahi. Awl-awl a anodoh diu ahi.
\end{itemize}

How does the attitude of the people of Joseph contrast with that of
Caleb?
\begin{itemize}
\item Joseph chilhah te chun a gamsung a um ho sakol kangtalai neiho chu
amu un, gamlah be ding angap pouve
\item Ahin, Caleb chun Pathen hatna a song in, kum somget le nga ahitah
vangin, gam lah ding ahol bebe nalai e
\end{itemize}

How does Joshua deal with their complaint (17:17-18)?
\begin{itemize}
\item Joshua in Joseph chilhah te kom a chun, mitamtah chuleh thahat na
hiuve, na jo diu ahi ati peh e
\end{itemize}

Seven tribes had not yet received their inheritance. What was their
problem (18:1-10)?
\begin{itemize}
\item Phung sagi ho chu gam alo theilou nailou u ahi. A mite chu a nodoh u
ahitan, kihom theilou uva, gam chu loding hita jongleh, alo theilou
u ahi.
\end{itemize}

As you look back over these chapters, what reasons can you give for
why Israel had difficulty taking full possession of the land?
\begin{itemize}
\item A gam a umte chu a vet uleh, a kichat jeh uvin, gam chu abon in alo
jou pouve.
\item A lo diu sa gam ho jong akihop theilou jeh uvin, alo gang thei pouve.
\end{itemize}

\subsection{Applying the Word}
\label{sec:org0c065e6}

Which reasons help to explain why God's promises to us are sometimes
only partially fulfilled? (For example, his promise to purify our
lives of sin.) Explain.
\begin{itemize}
\item Caleb bang in Pathen song leu hen, Pathen in eitep peh sau thenna
hinkho jong inei diu ahi. Galjou lou a ium uhi, lhepna hah jeh ingoh
nom uvin ahi. Ahin Pathen hatna song a, galjou ding ihiuve.
\end{itemize}

We sometimes act like the people of Joseph, complaining about how
little God has given us when we have not fully used what we have. How
do you think Joshua would respond to our complaints and excuses?
\begin{itemize}
\item Joseph phung ho bang a, ka chanding bang ka chang poi tia phunnoi
jong akihi nom e. Ahin Joshua in asei bang in, galjou thei cheh
ihiuve ti hi melchih jing a, galsat ding ahi. Pathen in galjona
eipeh diu ahi.
\end{itemize}

\subsection{Responding in Prayer}
\label{sec:org46e6e30}

Ask God to mold your faith to be like Caleb's.
\begin{itemize}
\item O Pathen, Caleb bang a nangma tahsan na a kingai ding in nei sem in
\item Nangma min a galjou ding kahi tihi gelpum a thanom tah a nangma
tepsa gamlo dingin kipan tange.
\end{itemize}
\section{June 09, 2021}
\label{sec:orgea510b8}

\textbf{\textbf{JOSHUA 10—12: THE LORD FIGHTS FOR ISRAEL}}

For more context before you begin studying, read this introduction to
the book of Joshua.

God demonstrates his faithfulness to every Christian. He strengthens
us when we are weak, comforts us when we are suffering, heals us when
we are sick. He provides for all of our physical, emotional and
spiritual needs. After all this, why do we often find it difficult to
trust him? Joshua and Israel had miraculously crossed the Jordan and
conquered the cities of Jericho and Ai. Yet in spite of these
victories, they still needed to be reassured that God was with them.

\subsection{Warming Up to God}
\label{sec:org7cb4256}

Why is it often hard to trust God for the future even though he has
been faithful in the past?

\href{https://www.biblegateway.com/passage/?search=Joshua\%2010-12\&version=NIV\&interface=print}{Read Joshua 10-12}

\subsection{Discovering the Word}
\label{sec:org4acc67f}

The Lord tells Joshua, "Do not be afraid of them; I have given them
into your hand" (10:8). After miraculously crossing the Jordan and
conquering Jericho and Ai,why would Joshua need this reassurance?

Verse 14 concludes, "Surely the LORD was fighting for Israel!" How is
this obvious from 10:9-15?

After the Lord reassures him, how does Joshua reassure his army about
future battles (10:16-27)?

How does the Lord demonstrate his faithfulness to Israel during the
southern campaign (10:29-43)?

During the northern campaign, how does Joshua demonstrate his
obedience to the Lord (11:6-23)?

In chapters 11—12 we do not read of any miraculous intervention by God
as we did in previous chapters. How did the Israelites know that God
was still the one giving them the victory?

\subsection{Applying the Word}
\label{sec:org110e566}

What can we do to encourage others about God's faithfulness?

Why is our obedience an important factor if we wish to see God's
power?

In spite of past victories, in what areas do you need to be reassured
of God's presence and power? Explain.

\subsection{Responding in Prayer}
\label{sec:orgf29d1f5}

Pray that God would make his presence real to you.

\section{June 08, 2021}
\label{sec:org9d50b56}
\textbf{\textbf{JOSHUA 9: DECEIVED}}

For more context before you begin studying, read this introduction to
the book of Joshua.

Satan's primary strategy is deceit. He seduces us into believing that
a lie is truth, that evil is good and that a "suicidal plunge is
really a leap into life" (Derek Kidner, Genesis [Downers Grove, IL:
InterVarsity Press, 1967], p. 68).

\subsection{Warming Up to God}
\label{sec:org1c36b47}

\begin{itemize}
\item How do you decide whether to make a decision on your own or to pray
about it first?
\begin{enumerate}
\item It is always safer to pray about decisions that we take in life
\item The bigger the decision, the more I would like to ask God about it
\item At times, it feels like indecision, but it is better to know we
are walking in the right path after having spent enough time in
prayer
\end{enumerate}

\href{https://www.biblegateway.com/passage/?search=Joshua\%209\&version=NIV\&interface=print}{Read Joshua 9}
\end{itemize}

\subsection{Discovering the Word}
\label{sec:org6149bbe}

\begin{itemize}
\item When they hear about Israel's victories, how does the Gibeonites'
reaction differ from that of the kings west of the Jordan (vv. 1-6)?
\begin{enumerate}
\item The other kings came up against the Israelites whereas the
Gibeonites came up with a deceitful plan to make peace with
Israel
\item They know the God of Israel
\end{enumerate}

\item What made the Gibeonites' deception so convincing to the Israelites
(vv. 7-13)?
\begin{enumerate}
\item They came with torn clothes, worn out sandals, dry and moldy
bread which made them appear as if they have traveled from a far
away country

\item The samples look convincing, they even asked them if they had
come from a nearby place perhaps. Their mistake was they didn't
equire of the Lord
\end{enumerate}

\item The Israelites were fooled because they "did not inquire of the
LORD" (v. 14). Why should they have known that this was not a
decision to be made on their own?
\begin{enumerate}
\item Making treaty with a nation is a big decision that too in the
Name of the Lord. It clearly requires of them to ask the Lord
\end{enumerate}

\item Why did the Israelites take their oath so seriously, even though it
was based on a lie (vv. 16-19)?
\begin{enumerate}
\item They have taken their oath in the Name of the Lord
\end{enumerate}

\item The Gibeonites are similar to Rahab in that, for both, their faith
led them to lie (vv. 22-27). Why aren't the Gibeonites commended for
their faith as Rahab was?
\begin{enumerate}
\item Gibeonites decieve Israel whereas Rahab help Israel in its plan
to conquer the land
\end{enumerate}
\end{itemize}

\subsection{Applying the Word}
\label{sec:orged5f774}

\begin{itemize}
\item What tricks does Satan use to keep us from seeking God's guidance?
\begin{enumerate}
\item By making us fear but God has not given us a spirit of fear.
\item Some well-meaning friends and family members will make things
appear as if it is God-ordained and we will not seek on our own
whether it is God's will
\end{enumerate}

\item In what areas are we tempted to make peace with a sinful world?
\begin{enumerate}
\item There are many areas\ldots{} in career decisions, etc
\end{enumerate}
\end{itemize}

\subsection{Responding in Prayer}
\label{sec:org8b6bd01}

Pray for those who are even now under Satan's deceitful influence.

\section{June 07, 2021}
\label{sec:orga8995f9}

\textbf{\textbf{JOSHUA 7—8: DEFEAT, CONFESSION AND VICTORY}}

For more context before you begin studying, read this introduction to
the book of Joshua.

Success can lead to complacency. We feel confident, in control,
optimistic—then suddenly the bottom drops out of our lives. Israel had
tasted success. They had entered the Promised Land, they had won an
important battle, and God was obviously with them. Conquering the city
of Ai would be a piece of cake! But their confidence collapsed when
they attacked the city and were routed. What had gone wrong? Had God
failed them? Why hadn't God kept his promise?

\subsection{Warming Up to God}
\label{sec:orgfd94abf}

Think of a time when you failed at something. How did you feel, and why?

\href{https://www.biblegateway.com/passage/?search=Joshua\%207-8\&version=NIV\&interface=print}{Read Joshua 7-8}

\subsection{Discovering the Word}
\label{sec:orgc32da58}

\begin{itemize}
\item What are Joshua's concerns after this defeat (7:6-9)?
\begin{enumerate}
\item He is concerned that Israel will be delivered to the enemies' hands
\item What then will happen to the Name of the Lord if they are wiped out
\end{enumerate}

\item How do God's concerns differ from Joshua's (7:10-15)?
\begin{enumerate}
\item God is concerned that Israel has violated the covenant they have
with him

\item They have stolen, they have lied\ldots{} That is why they cannot stand
against their enemies

\item God wants a people consecrated to him
\end{enumerate}

\item Joshua urged Achan to "give glory to the LORD" by admitting his
crime (7:19). How does confession glorify God?
\begin{enumerate}
\item Confession to God is admitting that we have sinned against him

\item By confessing we acknowledge he is God
\end{enumerate}

\item How did the second attack on Ai differ from the first (8:1-29)?
\begin{enumerate}
\item The enemy was completely defeated
\end{enumerate}

\item If you had been an Israelite, what thoughts would have come to mind
each time you saw the rock piles mentioned in 7:26 and 8:29?
\begin{enumerate}
\item It would remind them that they have to obey God and not break the
covenant

\item The other pile would remind them that God is with them and His
presence will help them conquer the Promised Land
\end{enumerate}

\item What impact would the reading of the law have had on the Israelites
(8:32-35)?
\begin{enumerate}
\item God wants to make sure that the law is heard and will be
remembered for all time to come since it has been carved in stone

\item Public reading of the law also reiterates it once again how
serious God is about his law
\end{enumerate}
\end{itemize}

\subsection{Applying the Word}
\label{sec:org34782f1}

\begin{itemize}
\item Joshua stated that Achan was being stoned because he had brought
disaster on the Israelites (7:25). Thirty-six Israelite warriors had
already died because of Achan's sin. How might our sins affect
others?
\begin{enumerate}
\item Our sins can affect not only us but some other innocent people
directly or indirectly
\item We need to be careful how we lead our lives, lest our actions can
affect others
\end{enumerate}

\item What reminders can help keep us from sinning?
\begin{enumerate}
\item By learning the word, having his words in our homes so that we
are always reminded

\item Take great care to know that God punishes sin. He wants a people
consecrated to him
\end{enumerate}
\end{itemize}

\subsection{Responding in Prayer}
\label{sec:org8821738}

\begin{itemize}
\item Bring any sin in your life before God. Confess it to him with a
penitent heart.
\end{itemize}

\section{June 05, 2021}
\label{sec:org0194a11}

\textbf{\textbf{JOHN 2: WINE \& A WHIP}}

For more context before you begin studying, read this introduction to the book
of John.

After I had given a presentation on the claims of Christ, a skeptical student
asked: "What proof do you have that Jesus really was who he claimed to be?"
People have been asking that question for two thousand years! For John the
convincing proof of Jesus’ deity was found in his words and deeds. No one but
God could say the things Jesus said, and no one but God could do the things
Jesus did. In this chapter are two signs that demonstrate that Jesus was the
fullness of God clothed in humanity.

\subsection{Warming Up to God}
\label{sec:org830ca3a}

Thank God for revealing himself to you personally.

\href{https://www.biblegateway.com/passage/?search=John\%202\&version=NIV\&interface=print}{Read John 2}

\subsection{Discovering the Word}
\label{sec:org0a0ce6b}

\begin{enumerate}
\item When the groom's parents ran out of wine for their guests, Jesus' mother
 asked him to help (v. 3). What do you think Mary expected Jesus to do?
(Remember, according to verse 11 Jesus had not yet performed any miracles.)
\begin{itemize}
\item Running out of wine would have been an embarassment for the host in this
case
\item Mary knew Jesus was a miracle child\ldots{} But did she really expect Jesus to
do a miracle? I dont think so.
\item Or, perhaps she already knew Jesus' ability to perform a miracle and that
is why she asked Jesus to help
\end{itemize}

\item What did Jesus mean by his reply to Mary in verse 4?
\begin{itemize}
\item His hour has not yet come according to Jesus

\item I dont understand the passage here, because Jesus anyway helped and turned
water into wine
\end{itemize}

\item According to verse 11, the purpose of Jesus' miracle was not to save the groom
from embarrassment but to display Christ's glory. What aspects of Christ's
glory does this miracle reveal to you?
\begin{itemize}
\item He can use simple things to do great things

\item To make it more personal, God save a sinner like me to be salt to the world
\end{itemize}

\item How does John's picture of Jesus in verses 15-16 fit with today's popular
concept of him?
\begin{itemize}
\item Jesus have zeal for the house of the Lord. In today's popular concept of
him, He was a great teacher who was not afraid to speak out against injustice
in the world.
\end{itemize}

\item Only the Messiah had the authority to cleanse the temple. The people
recognized that and asked Jesus for a miraculous sign to confirm his identity
(v. 18). To what "sign" did Jesus point them (vv. 19-22)? Why do you think
that particular sign was so significant in Jesus' mind?
\begin{itemize}
\item He alluded to his resurrection on the third day here. So from the beginning
of his ministry, he was clear of the path he was to take
\end{itemize}
\end{enumerate}

\subsection{Applying the Word}
\label{sec:orgcafe263}

\begin{enumerate}
\item In what practical ways can you demonstrate the same concern that Jesus does
toward the holy character of God?
\begin{itemize}
\item Just as Jesus was angry with people who take lightly the name of God and the
Lord's temple, we need to have holy anger against the wrongs we see in our
church, society and institutions
\end{itemize}

\item How do Jesus’ presence and actions at this party serve as a model for you?
\begin{itemize}
\item In the world we live in, we are to live in a way that reveal God's glory

\item We may not be able to do miracle just as Jesus did, but we are God's
representative and our concern is to bring glory to God in and through our
speech and actions
\end{itemize}
\end{enumerate}

\subsection{Responding in Prayer}
\label{sec:org4acfaef}

Ask God to help you to be his representative in everything you do.

\section{June 04, 2021}
\label{sec:org6d071ac}

\textbf{\textbf{JOHN 1: THE MASTER \& FIVE WHO FOLLOWED}}

For more context before you begin studying, read this introduction to the book of John. 

It was a great day in our history when a man first walked on the moon. But the Bible declares that a far greater event took place two thousand years ago. God walked on the earth in the person of Jesus Christ. John opens his Gospel with a beautiful hymn of exaltation to Christ. It is one of the most profound passages in all the Bible. It is written in simple, straightforward language, yet in studying the depths of its meaning, it is a passage where we never reach bottom. It is an ocean-sized truth, and we have to be content to paddle around in shallow water.

\subsection{Warming Up to God}
\label{sec:org72f7ea9}

Consider the miracle of God becoming human. Give him your praise and worship for what he has done for you.

Read John 1

\subsection{Discovering the Word}
\label{sec:org47065e3}

\begin{itemize}
\item John records more than a dozen names or descriptions of Jesus in this chapter. What are some of these?
\begin{enumerate}
\item the word
\item all things are created through him
\item the light
\item came into the world
\item he was Someone who John said was to come after him
\item the unique One
\end{enumerate}
\item In verses 1-3 what facts does John declare to be true of the Word?
\begin{enumerate}
\item already existed in the beginning
\item was with God
\item was God
\item existed in the beginning with God
\item God created everything through him
\end{enumerate}
\item According to verses 14-18, what specific aspects of God's character are revealed to us through Jesus?
\begin{enumerate}
\item full of unfailing love and faithfulness
\item glory
\item abundance
\end{enumerate}

\item What steps did John take to guarantee that people would not look at him but at Christ?
\begin{enumerate}
\item Someone was to come after him, he cried
\item He denied being the Messiah
\item He testified about Jesus as the Messiah
\end{enumerate}

\item In verses 35-51 we are introduced to five men: Andrew, Simon, Philip, Nathanael and one unnamed disciple (John). How did each man respond to the testimony he heard about Jesus?
\begin{enumerate}
\item Andrew searched for his brother Simon and introduced him to Jesus the Messiah

\item Philip told Nathanael that he found the Messiah

\item Nathanael declared to Jesus that he was the Messiah

\item John, the writer, wants to tell us upfront that Jesus is the Messiah
\end{enumerate}
\end{itemize}

\subsection{Applying the Word}
\label{sec:org8719e27}

\begin{itemize}
\item Which of the names of Jesus has the most significance to you personally? Explain why.
\end{itemize}


\begin{itemize}
\item What do you hope will happen in your life as a result of studying the Gospel of John?

\begin{enumerate}
\item Be more convinced of Jesus as the Messiah as written in the Scripture

\item See how God became man
\end{enumerate}
\end{itemize}

\subsection{Responding in Prayer}
\label{sec:org2303a31}

\begin{itemize}
\item Ask God to bring the light and life of Jesus to you as you study his Word.
\end{itemize}
\end{document}
